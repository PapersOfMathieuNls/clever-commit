\documentclass[sigconf]{acmart}
\usepackage{booktabs} % For formal tables
\usepackage{listings}
\usepackage[linesnumbered]{algorithm2e}
\usepackage[utf8]{inputenc}
\usepackage{multirow}
\usepackage{color}
\usepackage{pgfplots}
\usepackage{rotating}
\usepackage{pdflscape}
\usepackage{float}
\usepackage{amssymb}

\newcommand{\red}[1]{\textcolor{red}{#1}}
\newcommand{\wahab}[1]{\textcolor{blue}{\\ \\Wahab: #1}}
\newcommand{\mathieu}[1]{\textcolor{blue}{\\ \\Mathieu: #1}}

\usepackage[font=small,skip=2pt]{caption}

\usepackage{amssymb,amsmath}
\usepackage{ifxetex,ifluatex}
\usepackage{fixltx2e} % provides \textsubscript

\ifnum 0\ifxetex 1\fi\ifluatex 1\fi=0 % if pdftex
  \usepackage[T1]{fontenc}
  \usepackage[utf8]{inputenc}
\else % if luatex or xelatex
  \ifxetex
    \usepackage{mathspec}
    \usepackage{xltxtra,xunicode}
  \else
    \usepackage{fontspec}
  \fi
  \defaultfontfeatures{Mapping=tex-text,Scale=MatchLowercase}
  \newcommand{\euro}{€}


\fi
% use upquote if available, for straight quotes in verbatim environments
\IfFileExists{upquote.sty}{\usepackage{upquote}}{}
% use microtype if available
\IfFileExists{microtype.sty}{%
\usepackage{microtype}
\UseMicrotypeSet[protrusion]{basicmath} % disable protrusion for tt fonts
}{}


\hypersetup{breaklinks=true,
            bookmarks=true,
            pdfauthor={},
            pdftitle={CLEVER: Combining Code Metrics with Clone Detection for Just-In-Time Fault Prevention and Resolution in Large Industrial Projects},
            colorlinks=true,
            citecolor=blue,
            urlcolor=blue,
            linkcolor=magenta,
            pdfborder={0 0 0}}
\urlstyle{same}  % don't use monospace font for urls




\usepackage{color}
\usepackage{fancyvrb}
\newcommand{\VerbBar}{|}
\newcommand{\VERB}{\Verb[commandchars=\\\{\}]}
\DefineVerbatimEnvironment{Highlighting}{Verbatim}{commandchars=\\\{\}}
% Add ',fontsize=\small' for more characters per line
\newenvironment{Shaded}{}{}
\newcommand{\KeywordTok}[1]{\textcolor[rgb]{0.00,0.44,0.13}{\textbf{{#1}}}}
\newcommand{\DataTypeTok}[1]{\textcolor[rgb]{0.56,0.13,0.00}{{#1}}}
\newcommand{\DecValTok}[1]{\textcolor[rgb]{0.25,0.63,0.44}{{#1}}}
\newcommand{\BaseNTok}[1]{\textcolor[rgb]{0.25,0.63,0.44}{{#1}}}
\newcommand{\FloatTok}[1]{\textcolor[rgb]{0.25,0.63,0.44}{{#1}}}
\newcommand{\ConstantTok}[1]{\textcolor[rgb]{0.53,0.00,0.00}{{#1}}}
\newcommand{\CharTok}[1]{\textcolor[rgb]{0.25,0.44,0.63}{{#1}}}
\newcommand{\SpecialCharTok}[1]{\textcolor[rgb]{0.25,0.44,0.63}{{#1}}}
\newcommand{\StringTok}[1]{\textcolor[rgb]{0.25,0.44,0.63}{{#1}}}
\newcommand{\VerbatimStringTok}[1]{\textcolor[rgb]{0.25,0.44,0.63}{{#1}}}
\newcommand{\SpecialStringTok}[1]{\textcolor[rgb]{0.73,0.40,0.53}{{#1}}}
\newcommand{\ImportTok}[1]{{#1}}
\newcommand{\CommentTok}[1]{\textcolor[rgb]{0.38,0.63,0.69}{\textit{{#1}}}}
\newcommand{\DocumentationTok}[1]{\textcolor[rgb]{0.73,0.13,0.13}{\textit{{#1}}}}
\newcommand{\AnnotationTok}[1]{\textcolor[rgb]{0.38,0.63,0.69}{\textbf{\textit{{#1}}}}}
\newcommand{\CommentVarTok}[1]{\textcolor[rgb]{0.38,0.63,0.69}{\textbf{\textit{{#1}}}}}
\newcommand{\OtherTok}[1]{\textcolor[rgb]{0.00,0.44,0.13}{{#1}}}
\newcommand{\FunctionTok}[1]{\textcolor[rgb]{0.02,0.16,0.49}{{#1}}}
\newcommand{\VariableTok}[1]{\textcolor[rgb]{0.10,0.09,0.49}{{#1}}}
\newcommand{\ControlFlowTok}[1]{\textcolor[rgb]{0.00,0.44,0.13}{\textbf{{#1}}}}
\newcommand{\OperatorTok}[1]{\textcolor[rgb]{0.40,0.40,0.40}{{#1}}}
\newcommand{\BuiltInTok}[1]{{#1}}
\newcommand{\ExtensionTok}[1]{{#1}}
\newcommand{\PreprocessorTok}[1]{\textcolor[rgb]{0.74,0.48,0.00}{{#1}}}
\newcommand{\AttributeTok}[1]{\textcolor[rgb]{0.49,0.56,0.16}{{#1}}}
\newcommand{\RegionMarkerTok}[1]{{#1}}
\newcommand{\InformationTok}[1]{\textcolor[rgb]{0.38,0.63,0.69}{\textbf{\textit{{#1}}}}}
\newcommand{\WarningTok}[1]{\textcolor[rgb]{0.38,0.63,0.69}{\textbf{\textit{{#1}}}}}
\newcommand{\AlertTok}[1]{\textcolor[rgb]{1.00,0.00,0.00}{\textbf{{#1}}}}
\newcommand{\ErrorTok}[1]{\textcolor[rgb]{1.00,0.00,0.00}{\textbf{{#1}}}}
\newcommand{\NormalTok}[1]{{#1}}






\setlength{\emergencystretch}{3em}  % prevent overfull lines
\providecommand{\tightlist}{%
  \setlength{\itemsep}{0pt}\setlength{\parskip}{0pt}}
\setcounter{secnumdepth}{5}


\usepackage{booktabs}
\setcopyright{rightsretained}
\acmConference[ICSE'18]{ACM ICSE Conference}{May 2018}{Gothemburg, Sweden} 
\acmYear{018}
\copyrightyear{2018}

\acmArticle{4}
\acmPrice{15.00}

\begin{document}

\title{CLEVER: Combining Code Metrics with Clone Detection for Just-In-Time
Fault Prevention and Resolution in Large Industrial Projects}

\author{Double Blind Review}
\affiliation{%
  \institution{}
 \streetaddress{}
  \city{} 
  \state{} 
  \postcode{}
}
\email{}


\renewcommand{\shorttitle}{Combining Code Metrics With Clone Detection For Faults Prevention and Resolution}

\begin{abstract}
Automatic prevention and resolution of faults is an important research
topic in the field of software maintenance and evolution. Existing
approaches leverage code and process metrics to build statistical models
that can effectively prevent defect insertion and propose fixes in a
software project. Metrics, however, may vary from one project to
another, hindering the reuse of these models. Moreover, they tend to
generate high false positive rates by classifying healthy commits as
risky. Finally, they do not provide sufficient insights to developers on
how to fix the detected risky commits. In this paper, we propose an
approach, called CLEVER (Combining Levels of Bug Prevention and
Resolution techniques), that relies on a two-phases process for
intercepting risky commits before they reach the central repository. The
results show that CLEVER can detect risky commits with 79\% precision
and 65\% recall, which outperforms the performance of Commit-guru, a
well-known approach (66\% precision and 63\% recall) on the same
dataset. In addition, CLEVER is able to propose qualitative fixes to
transform risky commits into non-risky ones with a 66.7\% acceptance
rate.
\end{abstract}

%
% The code below should be generated by the tool at
% http://dl.acm.org/ccs.cfm
% Please copy and paste the code instead of the example below. 
%
\begin{CCSXML}
<ccs2012>
<concept>
<concept_id>10011007.10011074.10011099.10011102.10011103</concept_id>
<concept_desc>Software and its engineering~Software testing and debugging</concept_desc>
<concept_significance>500</concept_significance>
</concept>
<concept>
<concept_id>10011007.10011074.10011111.10011696</concept_id>
<concept_desc>Software and its engineering~Maintaining software</concept_desc>
<concept_significance>300</concept_significance>
</concept>
<concept>
<concept_id>10002951.10003227.10003241.10003243</concept_id>
<concept_desc>Information systems~Expert systems</concept_desc>
<concept_significance>100</concept_significance>
</concept>
</ccs2012>
\end{CCSXML}

\ccsdesc[500]{Software and its engineering~Software testing and debugging}
\ccsdesc[300]{Software and its engineering~Maintaining software}
\ccsdesc[100]{Information systems~Expert systems}

\keywords{Defect Predictions, Fault Fixing, Software Maintenance, Software Evolution}

\maketitle

\section{Introduction}\label{sec:introduction}

Automatic prevention and resolution of faults is an important research
topic in the field of software maintenance and evolution. A particular
line of research focuses on the problem of preventing the introduction
of faults by detecting risky commits (commits that may potentially
introduce faults in the system) before reaching the central code
repository. We refer to this as just-in-time fault detection/prevention.
There exist techniques that aim to detect risky commits (e.g., {[}2, 5,
44{]}), among which one recent approach is the one proposed Rosen et al.
{[}39{]}. The authors developed an approach and a supporting tool,
Commit-guru, that relies on building models from historical commits
using code and process metrics (e.g., code complexity, the experience of
the developers, etc.) as the main features. These models are used to
classify new commits as risky or not. Commit-guru has been shown to
outperform previous techniques (e.g., {[}18, 25{]}).

Commit-guru and similar tools suffer from a number of limitations.
First, they tend to generate high false positive rates by classifying
healthy commits as risky. The second limitation is that they do not
provide sufficient insights to developers on how to fix the detected
risky commits. They simply return measurements that are often difficult
to interpret by developers. In addition, they have been mainly validated
using open source systems. Their effectiveness when applied to
industrial systems has yet to be shown.

In this paper, we propose an approach, called CLEVER (Combining Levels
of Bug Prevention and Resolution techniques), that relies on a
two-phases process for intercepting risky commits before they reach the
central repository. The first phase consists of building a metric-based
model to assess the likelihood that an incoming commit is risky or not.
This is similar to existing approaches. The next phase relies on clone
detection to compare code blocks extracted from risky commits (detected
in the first phase) with those of known historical fault-introducing
commits. This additional phase provides CLEVER with two apparent
advantages over Commit-guru. First, as we will show in the evaluation
section, CLEVER is able to reduce the number of false positives by
relying on code matching instead of mere metrics. The second advantage
is that, with CLEVER, it is possible to use commits that were used to
fix faults introduced by previous commits to guide the developers on how
to improve the risky commits at hand. This way, CLEVER goes one step
further than Commit-guru (and similar techniques) by providing
developers with a potential fix for their risky commits.

Another important aspect of CLEVER is its ability to detect risky
commits not only by comparing them to commits of a single project but
also to those belonging to other projects that share common
dependencies. This is important in the context of an industrial project
where software systems tend to have many dependencies that make them
vulnerable to the same faults.

CLEVER was developed in collaboration with software developers from
Ubisoft La Forge. Ubisoft is one of the world's largest video game
development companies specializing in the design and implementation of
high-budget video games. Ubisoft software systems are highly coupled
containing millions of files and commits, developed and maintained by
more than 8,000 developers scattered across 29 locations in six
continents.

We tested CLEVER on 12 major Ubisoft systems. The results show that
CLEVER can detect risky commits with 79\% precision and 65\% recall,
which outperforms the performance of Commit-guru (66\% precision and
63\% recall) when applied to the same dataset. In addition, 66.7\% of
the proposed fixes were accepted by a least one Ubisoft software
developer, making CLEVER an effective and practical approach for the
detection and resolution of risky commits.

The remaining parts of this paper are organised as follows. In Section
\ref{sec:relwork}, we present related work. Sections \ref{sec:CLEVERT},
\ref{sec:exp} and \ref{sec:result} are dedicated to describing the
CLEVER approach, the case study setup, and the case study results. Then,
Sections \ref{sec:threats} and \ref{sec:conclusion} present the threats
to validity and a conclusion accompanied with future work.

\section{Related Work}\label{sec:relwork}

Our approach, CLEVER, is related to two research areas: defect
prediction and patch generation.

\subsection{File, Module and Risky Change
Prediction}\label{file-module-and-risky-change-prediction}

Existing studies for predicting risky changes within a repository rely
mainly on code and process metrics. As discussed in the introduction
section, Rosen et al. {[}39{]} developed Commit-guru a tool that relies
on building models from historical commits using code and process
metrics (e.g., code complexity, the experience of the developers, etc.)
as the main features. There exist other studies that leverage several
code metric suites such as the CK metrics suite {[}5{]} or the Briand's
coupling metrics {[}2{]}. These metrics have been used, with success, to
predict defects as shown by Subramanyam \emph{et al.} {[}44{]} and
Gyimothy \emph{et al.} {[}11{]}.

Further improvements to these metrics have been proposed by Nagappan
\emph{et al.} {[}31, 33{]} and Zimmerman \emph{et al.} {[}48, 49{]} who
used call graphs as the main artifact for computing code metrics with a
static analyzer.

Nagappan \emph{et al.} et proposed a technique that uses data mined from
source code repository such as churns to assess the quality of a change
{[}32{]}. Hassan \emph{et al} and Ostrand \emph{et al} used past changes
and defects to predict buggy locations {[}12{]}, {[}36{]}. Their methods
rely on various heuristics to identify the locations that are most
likely to introduce a defect. Kim \emph{et al} {[}24{]} proposed the bug
cache approach, which is an improved technique over Hassan and Holt's
approach {[}12{]}. Rahman and Devanbu found that, in general,
process-based metrics perform as good as code-based metrics {[}38{]}.

Other studies that aim to predict risky changes use the entropy of a
given change {[}13{]} and the size of the change combined with files
being changed {[}19{]}.

These techniques operate at different levels of the systems and may
require the presence of the entire source code. In addition, the
reliance of metrics may result in high false positives rates. We need a
way to to validate whether a suspicious change is indeed risky. In this
paper, we address this issue using a two-phase process that combines the
use of metrics to detect suspicious risky changes, and code matching to
increase the detect accuracy. As we will show in the evaluation section,
CLEVER reduces the number of false positives while keeping good recall.
In addition, CLEVER operates at commit-time for preventing the
introduction of faults before they reach the code repository. Through
interactions with Ubisoft developers, we found that this integrates well
with the workflow of developers.

\subsection{Automatic Patch
Generation}\label{automatic-patch-generation}

One feature of CLEVER is the ability to propose fixes that can help
developers correct the detected risky commit. This is similar in
principle to the work on automatic patch generation. Pan \emph{et al.}
and Kim \emph{et al.} proposed two approaches that extract and apply fix
patterns {[}22, 37{]}. Pan \emph{et al.} identified 27 patterns and were
able to fix 45.7\% - 63.6\% of bugs using one of the proposed patterns.
The patterns found by Kim \emph{et al.} are mined from human-written
patches and were able to successfully generate patches for 27 out of 119
bugs. The tool by Kim \emph{et al.}, named PAR, is similar to the second
part of CLEVER where we propose fixes. Our approach also mines potential
fixes from human-written patches found in the historical data. In our
work, we do not generate patches, but instead propose known patches to
developers for further assessment. It has also been shown that patch
generation is useful in understanding and debugging the causes of faults
{[}46{]}.

Despite the advances in the field of automatic patch generation, this
task remains overly complex. Developers expect from tools high quality
patches that can be safely deployed. Many studies proposed a
classification of what is considered an acceptable quality patch for an
automatically generated patch to be adopted in industry {[}7, 26, 27{]}.

\section{The CLEVER Approach}\label{sec:CLEVERT}

Figures \ref{fig:CLEVERT1}, \ref{fig:CLEVERT3} and \ref{fig:CLEVERT2}
show an overview of the CLEVER approach, which consists of two parallel
processes.

In the first process (Figures \ref{fig:CLEVERT1} and
\ref{fig:CLEVERT3}), CLEVER manages events happening on project tracking
systems to extract fault-introducing commits and commits and their
corresponding fixes. For simplicity reasons, in the rest of this paper,
we refer to commits that are used to fix defects as \emph{fix-commits}.
We use the term \emph{defect-commit} to mean a commit that introduces a
fault.

The project tracking component of CLEVER listens to bug (or issue)
closing events of Ubisoft projects. Currently, CLEVER is tested with 1
large project within Ubisoft and have been evaluated with 11 other large
projects. These projects share many dependencies. We clustered them
based on their dependencies with the aim to improve the accuracy of
CLEVER. This clustering step is important in order to identify faults
that may exist due to dependencies, while enhancing the quality of the
proposed fixes. Applying CLEVER to projects that are not related to each
other is ineffective as shown in our experiments.

\begin{figure*}
  \centering
    \includegraphics[width=\textwidth]{media/fix-approach.png}
    \caption{Managing events happening on project tracking systems to extract defect-introducing commits and commits that provided the fixes\label{fig:bianca1}}
\end{figure*}

\begin{figure*}
  \centering
    \includegraphics[width=0.7\textwidth]{media/cluster-approach}
    \caption{Clustering by dependency\label{fig:bianca3}}
\end{figure*}

\begin{figure*}
  \centering
    \includegraphics[width=\textwidth]{media/detect-approach}
    \caption{Classifying incoming commits and proposing fixes\label{fig:bianca2}}
\end{figure*}



In the second process (Figure \ref{fig:CLEVERT2}), CLEVER intercepts
incoming commits before they leave developers' workstations using the
concept of pre-commit hooks. A pre-commit hook is a script that is
executed at commit-time and it is supported by most major code
versioning systems such as \emph{Git}. There are two types of hooks:
client-side and server-side. Client-side hooks are triggered by
operations such as committing and merging, whereas server-side hooks run
on network operations such as receiving pushed commits. These hooks can
be used for different purposes such as checking compliance with coding
rules, or the automatic execution of unit tests. A pre-commit hook runs
before a developer specifies a commit message.

Ubisoft's developers use pre-commit hooks for all sorts of reasons such
as identifying the tasks that are addressed by the commit at hand,
specifying the reviewers who will review the commit, and so on.
Implementing this part of CLEVER as a pre-commit hook is an important
step towards the integration of CLEVER with the workflow of developers
at Ubisoft. The developers do not have to download, install, and
understand additional tools in order to use CLEVER.

Once the commit is intercepted, we compute code and process metrics
associated with this commit. The selected metrics are discussed further
in Section \ref{sec:offline}. The result is a feature vector (Step 4)
that is used for classifying the commit as \emph{risky} or
\emph{non-risky}.

If the commit is classified as \emph{non-risky}, then the process stops,
and the commit can be transferred from the developer's workstation to
the central repository. \emph{Risky} commits, on the other hand, are
further analysed in order to reduce the number of false positives
(healthy commits that are detected as risky). We achieve this by first
extracting the code blocks that are modified by the developer and then
compare them to code blocks of known fault-introducing commits.

\subsection{Clustering Projects}\label{sec:clustering}

We cluster projects according to their dependencies. The rationale is
that projects that share dependencies are most likely to contain defects
caused by misuse of these dependencies. In this step, the project
dependencies are analysed and saved into a single NoSQL graph database
as shown in Figure \ref{fig:CLEVERT3}. A node corresponds to a project
that is connected to other projects on which it depends. Dependencies
can be \emph{external} or \emph{internal} depending on whether the
products are created in-house or supplied by a third-party. For
confidentiality reasons, we cannot reveal the name of the projects
involved in the project dependency graph. We show the 12 projects in
yellow color with their dependencies in blue color in Figure
\ref{fig:dep-graph}. In total, we discovered 405 distinct dependencies
that internal and external both. The resulting partitioning is shown in
Figure \ref{fig:network-sample}.

\begin{figure*}
  \centering
    \includegraphics[width=0.70\textwidth]{media/network.png}
    \caption{Dependency Graph\label{fig:dep-graph}}
\end{figure*}

Internal dependencies are managed within the framework of a single
repository, which makes their automatic extraction possible. The
dependencies could also be automatically retrieved if the projects use a
dependency manager such as Maven.

\begin{figure}
  \centering
    \includegraphics[width=0.40\textwidth]{media/network-sample.png}
    \caption{Clusters\label{fig:network-sample}}
\end{figure}

Once the project dependency graph is extracted, we use a clustering
algorithm to partition the graph. To this end, we choose the
Girvan--Newman algorithm {[}10, 35{]}, used to detect communities by
progressively removing edges from the original network. Instead of
trying to construct a measure that identifies the edges that are the
most central to communities, the Girvan--Newman algorithm focuses on
edges that are most likely ``between'' communities. This algorithm is
very effective at discovering community structure in both
computer-generated and real-world network data {[}35{]}. Other
clustering algorithms can also be used.

\subsection{Building a Database of Code Blocks of Defect-Commits and
Fix-Commits}\label{sec:offline}

To build our database of code blocks that are related to defect-commits
and fix-commits, we first need to identify the respective commits. Then,
we extract the relevant blocks of code from the commits.

\textbf{Extracting Commits:} CLEVER listens to issue closing events
happening on the project tracking system used at Ubisoft. Every time an
issue is closed, CLEVER retrieves the commit that was used to fix the
issue (the fix-commit) as well as the one that introduced the defect
(the defect-commit). To link fix-commits and their related issues we
implemented the well-known SZZ algorithm presented by Kim et al.
{[}23{]}.

\textbf{Extracting Code Blocks:} Algorithm \ref{alg:extract} presents an
overview of how to extract blocks. This algorithm receives as arguments,
the changesets and the blocks that have been previously extracted. Then,
Lines 1 to 5 show the \(for\) loop that iterates over the changesets.
For each changeset (Line 2), we extract the blocks by calling the
\(~extract\_blocks(Changeset~cs)\) function. In this function, we expand
our changeset to the left and to the right in order to have a complete
block.

\begin{algorithm}
 \KwData{$Changeset[]$ changesets\;
 $Block[]$ prior\_blocks\;
 }
 \KwResult{Up to date blocks of the systems}
 \For{$i \leftarrow 0$ \KwTo$size\_of~changesets$}{
    Block[] blocks $\leftarrow$ $extract\_blocks(changesets)$\;
    \For{$j \leftarrow 0$ \KwTo$size\_of~blocks$}{
       write $blocks[j]$\;
    }
 }

 \SetKwProg{myproc}{Function}{ $~extract\_blocks(Changeset~cs)$}{}
   \myproc{{}}{

   \uIf{$cs~is~unbalanced~right$}{$cs \leftarrow expand\_left(cs)$\;}

   \ElseIf{$cs~is~unbalanced~left$}{$cs \leftarrow expand\_right(cs)$\;}

   \nl\KwRet$txl\_extract\_blocks(cs)$\;
   }


 \caption{Overview of the Extract Blocks Operation\label{alg:extract}}
\end{algorithm}

As depicted by the diff below (not from Ubisoft), changesets contain
only the modified chunk of code and not necessarily complete blocks.

\begin{Shaded}
\begin{Highlighting}[]
\DataTypeTok{@@ -315,36 +315,6 @@}
\NormalTok{int initprocesstree_sysdep}
\NormalTok{(ProcessTree_T **reference) \{}
    \NormalTok{mach_port_deallocate(mytask,}
      \NormalTok{task);}
\NormalTok{\}}
\NormalTok{\}}
\StringTok{- if (task_for_pid(mytask, pt[i].pid,}
\StringTok{-  &task) == KERN_SUCCESS) \{}
\StringTok{-   mach_msg_type_number_t   count;}
\StringTok{-   task_basic_info_data_t   taskinfo;}
\end{Highlighting}
\end{Shaded}

Therefore, we need to expand the changeset to the left (or right) to
have syntactically correct blocks. We do so by checking the block's
beginning and ending with parentheses algorithms {[}3{]}.

\subsection{Building a Metric-Based Model}\label{sec:metric-based}

We adapted Commit-guru {[}39{]} for building the metric-based model.
Commit-guru uses a list of keywords proposed by Hindle \emph{et al.}
{[}14{]} to classify commit in terms of \emph{maintenance},
\emph{feature} or \emph{fix}. Then, it uses the SZZ algorithm to find
the defect-commit linked to the fix-commit. For each defect-commit,
Commit-guru computes the following code metrics: \emph{la} (lines
added), \emph{ld} (lines deleted), \emph{nf} (number of modified files),
\emph{ns} (number of modified subsystems), \emph{nd} (number of modified
directories), \emph{en} (distriubtion of modified code across each
file), \emph{lt} (lines of code in each file (sum) before the commit),
\emph{ndev} (the number of developers that modifed the files in a
commit), \emph{age} (the average time interval between the last and
current change), \emph{exp} (number of changes previously made by the
author ), \emph{rexp} (experience weighted by age of files (1 / (n +
1))), \emph{sexp} (previous changes made by the author in the same
subsystem), \emph{loc} (total number of modified LOC across all files),
\emph{nuc} (number of unique changes to the files). Then, a statistical
model is built using the metric values of the defect-commits. Using
linear regression, Commit-guru is able to predict whether incoming
commits are \emph{risky} or not.

We had to modify Commit-guru to fit the context of this study. First, we
used information found in Ubisoft's internal project tracking system
used to classify the purpose of a commit (i.e., \emph{maintenance},
\emph{feature} or \emph{fix}). In other words, CLEVER only classifies a
commit as a defect-commit if it is the root cause of a fix linked to a
crash in the internal project tracking system. Using internal pre-commit
hooks, Ubisoft developers must link every commit to a given task \#ID.
If the task \#ID entered by the developer matches a bug or crash report
within the project tracking system, then we perform the SCM
blame/annotate function on all the modified lines of code for their
corresponding files on the fix-commit's parents. This returns the
commits that previously modified these lines of code, and are flagged as
defect-commits. Another modification consists of the actual
classification algorithm. We did not use linear regression but instead
the random forest algorithm. The random forest algorithm turned out to
be more effective as described in Section \ref{sec:result}. Finally, we
had to rewrite Commit-guru in GoLang for performance and internal
reasons.

\subsection{Comparing Code Blocks}\label{sec:online}

Each time a developer makes a commit, CLEVER intercepts it using a
pre-commit hook and classifies it as \emph{risky} or not. If the commit
is classified as \emph{risky} by the metric-based classifier, then, we
extract the corresponding code block (in a similar way as in the
previous phase), and compare it to the code blocks of historical
defect-commits. If there is a match, then the new commit is deemed to be
risky. A threshold \(\alpha\) is used to assess the extent beyond which
two commits are considered similar.

To compare the extracted blocks to the ones in the database, we resort
to clone detection techniques, more specifically, text-based clone
detection techniques. This is because lexical and syntactic analysis
approaches (alternatives to text-based comparisons) would require a
complete program to work, i.e., a program that compiles. In the
relatively wide-range of tools and techniques that exist to detect
clones by considering code as text {[}8, 16, 17, 29, 30, 47{]}, we chose
the NICAD clone detector because it is freely available and has shown to
perform well {[}6{]}. We improved NICAD to process blocks that comes
from commit-diffs. This is because the current version of NICADcan only
process syntactically correct code and commit-diffs are, by definition,
snippets that represent modified regions of a given set of files.

By reusing NICAD, CLEVER can detect Types 3 software clones {[}21{]}.
Type 3 clones can contain added or deleted code statements, which make
them suitable for comparing commit code blocks. In addition, NICAD uses
a pretty-printing strategy from where statements are broken down into
several lines {[}40{]}. This functionality allowed us to detect Segments
1 and 2 as a clone pair, as shown by Table \ref{tab:pretty-printing},
because only the initialization of \(i\) changed. This specific example
would not have been marked as a clone by other tools we tested such as
Duploc {[}8{]}.

\begin{table*}[]
\centering
\caption{Pretty-Printing Example}
\label{tab:pretty-printing}
\resizebox{0.5\textwidth}{!}{%
\begin{tabular}{l|l|l|l|l|l}
\hline
Segment 1          & Segment 2          & Segment 3           & S1 \& S2 & S1 \& S3 & S2 \& S3 \\ \hline \hline
for (              & for (              & for (               & 1        & 1        & 1        \\
i = 0;             & i = 1;             & j = 2;              & 0        & 0        & 0        \\
i \textgreater 10; & i \textgreater 10; & j \textgreater 100; & 1        & 0        & 0        \\ 
i++)               & i++)               & j++)                & 1        & 0        & 0        \\ \hline \hline
\multicolumn{3}{c|}{Total Matches}                            & 3        & 1        & 1        \\ \hline
\multicolumn{3}{c|}{Total Mismatches}                         & 1        & 3        & 3 \\ \hline

\end{tabular}
}
\end{table*}

The extracted, pretty-printed, normalized filtered blocks are marked as
potential clones using a Longest Common Subsequence (LCS) algorithm
{[}15{]}. Then, a percentage of unique statements can be computed and,
given the threshold \(\alpha\), the blocks are marked as clones.

\subsection{Classifying Incoming
Commits}\label{classifying-incoming-commits}

As discusses in Section \ref{sec:metric-based}, a new commit goes
through the metric-based model first (Steps 1 to 4). If the commit is
classified as \emph{non-risky}, we simply let it through, and we stop
the process. If the commit is classified as \emph{risky}, however, we
continue the process with Steps 5 to 9 our approach.

One may wonder why we needed to have a metric-based model in the first
place. We could have resorted to clone detection as the main mechanism.
The main reason for having the metric-based model is efficiency. If each
commit had to be analysed against all known signatures using code clone
similarity, then, it would have made CLEVER time consuming. We estimate
that, in an average workday, if all commits had to be compared against
all signatures on our cluster it would take around 25 minutes to process
a commit. In comparison, it takes, in average, 3.75 seconds with the
current approach.

\subsection{Proposing Fixes}\label{proposing-fixes}

An important aspect in the design of CLEVER is the ability to provide
guidance to developers on how to improve risky commits. We achieve this
by extracting from the database the fix-commit corresponding to the top
1 matching defect-commit and present it to the developer. We believe
that this makes CLEVER a practical approach. Developers can understand
why a given modification has been reported as risky by looking at code
instead of simple metrics as in the case of the studies reported in
{[}18, 39{]}.

Finally, using the fixes of past defects, we can provide a solution, in
the form of a contextualised diff, to the developers. A contextualised
diff is a diff that is modified to match the current workspace of the
developer regarding variable types and names. In Step 8 of Figure 3, we
adapt the matching fixes to the actual context of the developer by
modifying indentation depth and variable name in an effort to reduce
context switching. We believe that this would make it easier for
developers to understand the proposed fixes and see if it applies in
their situation.

\section{Case Study Setup}\label{sec:exp}

In this section, we present the setup of our case study in terms of
repository selection, dependency analysis, comparison process and
evaluation measures.

\subsection{Project Repository Selection}\label{sec:rep}

In collaboration with Ubisoft developers, we selected 12 major software
projects (i.e., systems) developed at Ubisoft to evaluate the
effectiveness of CLEVER. These systems continue to be actively
maintained by thousands of developers. Ubisoft projects are organized by
game engines. A game engine can be used in the development of many
high-budget games. The projects selected for this case study are related
to the same game engine. For confidentiality and security reasons,
neither the names nor the characteristics of these projects are
provided. We can however disclose that the size of these systems
altogether consists of millions of lines of code.

\subsection{Project Dependency Analysis}\label{sec:dependencies}

Figure \ref{fig:dep-graph} shows the project dependency graph. As shown
in Figure \ref{fig:dep-graph}, these projects are highly interconnected.
A review of each cluster shows that this partitioning divides projects
in terms of their high-level functionalities. For example, one cluster
is related to a particular given family of video games, whereas the
other cluster refers to another family. We showed this partitioning to
11 experienced software developers and ask them to validate it. They all
agreed that the results of this automatic clustering is accurate and
reflects well the various projects groups of the company.

\subsection{Building a Database of Defect-Commits and
Fix-Commits}\label{sub:golden}

To build the database that we can use to assess the performance of
CLEVER, we use the same process as discussed in Section
\ref{sec:offline}. We retrieve the full history of each project and
label commits as defect-commits if they appear to be linked to a closed
issue using the SZZ algorithm {[}23{]}. This baseline is used to compute
the precision and recall of CLEVER. Each time CLEVER classifies a commit
as \emph{risky}; we can check if the \emph{risky} commit is in the
database of defect-introducing commits. The same evaluation process is
used by related studies {[}1, 9, 19, 25, 28{]}.

\subsection{Process of Comparing New Commits}\label{sec:newcommits}

Because our approach relies on commit pre-hooks to detect risky commits,
we had to find a way to \emph{replay} past commits. To do so, we
\emph{cloned} our test subjects, and then created a new branch called
\emph{CLEVER}. When created, this branch is reinitialized at the initial
state of the project (the first commit), and each commit can be replayed
as they have originally been. For each commit, we store the time taken
for \emph{CLEVER} to run, the number of detected clone pairs, and the
commits that match the current commit. As an example, suppose that we
have three commits from two projects. At time \(t_1\), commit \(c_1\) in
project \(p_1\) introduces a defect. The defect is experienced by a user
that reports it via an issue \(i_1\) at \(t_2\). A developer fixes the
defect introduced by \(c_1\) in commit \(c_2\) and closes \(i_1\) at
\(t_3\). From \(t_3\) we known that \(c_1\) introduced a defect using
the process described in Section \ref{sub:golden}. If at \(t_4\),
\(c_3\) is pushed to \(p_2\) and classify by the metric-based classifier
as \emph{risky}, we extract \(c_3\) blocks and compares them with the
ones of \(c_1\). If \(c3\) and \(c1\) are a match after preprocessing,
pretty-printing and formatting, then \(c_3\) is classified as
\emph{risky} by CLEVER and \(c_2\) is proposed to the developer as a
potential solution for the defect introduced in \(c_3\).

\subsection{Evaluation Measures}\label{evaluation-measures}

Similar to prior work (e.g., {[}19, 45{]}), we used precision, recall,
and F\(_1\)-measure to evaluate our approach. They are computed using TP
(true positives), FP (false positives), FN (false negatives), which are
defined as follows:

\begin{itemize}
\tightlist
\item
  TP is the number of defect-commits that were properly classified by
  CLEVER
\item
  FP is the number of healthy commits that were classified by CLEVER as
  risky
\item
  FN is the number of defect introducing-commits that were not detected
  by CLEVER
\item
  Precision: TP / (TP + FP)
\item
  Recall: TP / (TP + FN)
\item
  F\(_1\)-measure: 2.(precision.recall)/(precision+recall)
\end{itemize}

It is worth mentioning that, in the case of defect prevention, false
positives can be hard to identify as the defects could be in the code
but not yet reported through a bug report (or issue). To address this,
we did not include the last six months of history. Following similar
studies {[}4, 20, 39, 43{]}, if a defect is not reported within six
months then it is not considered.

\section{Case Study Results}\label{sec:result}

In this section, we show the effectiveness of CLEVER in detecting risky
commits using a combination of metric-based models and clone detection.
The main research question addressed by this case study is: \emph{Can we
detect risky commits by combining metrics and code comparison within and
across related Ubisoft projects, and if so, what would be the accuracy?}

The experiments took nearly two months using a cluster of six 12 3.6 Ghz
cores with 32GB of RAM each. The most time consuming part of the
experiment consists of building the baseline as each commit must be
analysed with the SZZ algorithm. Once the baseline was established, the
model built, it took, on average, 3.75 seconds to analyse an incoming
commit on our cluster.

In the following subsections, we provide insights on the performance of
CLEVER by comparing it to Commit-guru {[}39{]} alone, i.e., an approach
that relies only on metric-based models. We chose Commit-guru because it
has been shown to outperform other techniques (e.g., {[}18, 25{]}).
Commit-guru is also open source and easy to use.

\subsection{Performance of CLEVER}\label{performance-of-clever}

when applied to 12 Ubisoft projects, CLEVER detects risky commits with
an average precision, recall, and F1-measure of 79.10\%, a 65.61\%, and
71.72\% respectively. For clone detection, we used a threshold of 30\% .
This is because Roy \emph{et al.} {[}41{]} showed through empirical
studies that using NICAD with a threshold of around 30\%, the default
setting, provides good results for the detection of Type 3 clones. When
applied to the same projects, Commit-guru achieves an average precision,
recall, and F1-measure of 66.71\%, 63.01\% and 64.80\%, respectively.

We can see that the second phase of CLEVER (clone detection)
considerably reduces the number of false positives (precision of 79.10\%
for CLEVER compared to 66.71\% for Commit-guru), while achieving similar
recall (65.61\% for CELEVER compared to 63.01\% for Commit-guru).

\subsection{Cluster Classifier
Performance}\label{cluster-classifier-performance}

The clusters computed with the dependencies of each project help to
solve an important problem in defect prediction, known as cold start
{[}42{]}. Indeed, as performant as a classifier can be it, after
training, it still needs to be train with historical data. While the
system is learning, no prediction can be done.

There exist approaches have been proposed to solve this problems,
however, they all require to classify \emph{similar} projects by hand
and then, manipulate the feature space in order to adapt the model
learnt from one project to another one {[}34{]}.

With our approach, the \emph{similarity} between projects is computed
automatically and the results show that we do not actually need to
manipulate the feature space. Figures \ref{fig:bluecluster},
\ref{fig:yellowcluster} and \ref{fig:redcluster} show the performance of
CLEVER classification in terms of ROC-curve for the first thousand
commits of the last project (chronologically) for the blue, yellow and
red clusters presented in Figure \ref{fig:network-sample}. The left
sides or the graph are low cutoff (aggressive) while the right sides are
high cutoff (conservative). The area under the ROC curves are 0.817,
0.763 and 0.806 for the blue, yellow and red clusters, respectively. In
other words, if we train a classifier with historical data from system
in the same cluster as the targeted systems, then we do not have to wait
to start classifying incoming commits. To confirm this, we ran an
experiment with the blue cluster where we first apply the model learnt
from other members of the cluster for the first thousand commits. The
performance of this model is 75.1\% precision, 57.6\% recall for the
first thousand commits. To evaluate the added performance obtained via
the cluster, we ran the same experiment but using a model built with
data from the red cluster on the last project of the blue cluster. The
results are shown in Figure \ref{fig:redonblue}. While the AUC for the
recall versus the specificity stays in acceptable range at 0.708, we see
a drastic decrease in the obtained precision. Indeed, the precision
never surpasses 30\% (for a recall of 3\%).

\begin{figure}
  \centering
    \includegraphics[width=0.5\textwidth]{media/bluecluster.png}
    \caption{Performances of Misfire while cold-starting the last project in the blue cluster\label{fig:bluecluster}}
\end{figure}

\begin{figure}
  \centering
    \includegraphics[width=0.5\textwidth]{media/yellowcluster.png}
    \caption{Performances of Misfire while cold-starting the last project in the yellow cluster\label{fig:yellowcluster}}
\end{figure}


\begin{figure}
  \centering
    \includegraphics[width=0.5\textwidth]{media/redcluster.png}
    \caption{Performances of Misfire while cold-starting the last project in the red cluster\label{fig:redcluster}}
\end{figure} \begin{table*}[]
\centering
\caption{Workshop results}
\label{tab:Workshop}
\resizebox{\textwidth}{!}{%
\begin{tabular}{c|c|c|c|c|c|c|c|c|c|c|c|c}
   & F1        & F2 & F3        & F4        & F5 & F6        & F7 & F8        & F9 & F10       & F11       & F12 \\ \hline
P1 & Accepted & Rejected   & Accepted & Accepted & Unsure  & Accepted & Unsure  & Rejected          & Rejected   & Accepted & Accepted & Unsure   \\
P2 & Accepted & Rejected   & Accepted & Unsure         & Unsure  & Accepted & Unsure  & Rejected          & Rejected   & Accepted & Accepted & Unsure   \\
P3 & Accepted & Rejected   & Accepted & Unsure         & Unsure  & Accepted & Unsure  & Rejected          & Rejected   & Accepted & Accepted & Unsure   \\
P4 & Accepted & Rejected   & Accepted & Unsure         & Unsure  & Accepted & Unsure  & Accepted & Rejected   & Accepted & Accepted & Unsure   \\
P5 & Accepted & Rejected   & Accepted & Accepted & Unsure  & Accepted & Unsure  & Rejected          & Rejected   & Accepted & Accepted & Unsure   \\
P6 & Accepted & Rejected   & Accepted & Unsure         & Unsure  & Accepted & Unsure  & Accepted & Rejected   & Accepted & Accepted & Unsure   \\ \hline
\end{tabular}
}
\end{table*}

\subsection{Analysis of the Quality of the Fixes Proposed by
CLEVER}\label{analysis-of-the-quality-of-the-fixes-proposed-by-clever}

In order to validate the quality of the fixes proposed by CLEVER, we
conducted an internal workshop where we invited a number of Ubisoft
development team. The workshop was attended by six participants: two
software architects, two developers, one technical lead, and one IT
project manager. The participants have many years of experience at
Ubisoft.

The participants were asked to review 12 randomly selected fixes that
were proposed by CLEVER. These fixes are related to one system in which
the participants have excellent knowledge. We presented them with the
original buggy commits, the original fixes for these commits, and the
fixes that were automatically extracted by CLEVER. We asked them the
following questions \emph{``Is the proposed fix applicable in the given
situation?''} for each fix.

The review session took around 60 minutes. This does not include the
time it took to explain the objective of the session, the setup, the
collection of their feedback, etc.

We asked the participants to rank each fix that is proposed by CLEVER
using this scheme:

\begin{itemize}
\tightlist
\item
  Fix Accepted: The participant found the fix proposed by CLEVER
  applicable to the risky commit.
\item
  Unsure: In this situation, the participant is unsure about the
  relevance of the fix. There might be a need for more information to
  arrive to a verdict.
\item
  Fix Rejected: The participant found the fix is not applicable to the
  risky commit.
\end{itemize}

Table \ref{tab:Workshop} shows answers of the participants. The columns
refer to the fixes proposed by CLEVER, whereas the rows refer to the
participants that we denote using P1, P2, \ldots{}, P6. As we can see
from the table, 41.6\% of the proposed fixes (F1, F3, F6, F10 and F12)
have been accepted by all participants, while 25\% have been accepted by
at least one member (F4, F8, F11). We analysed the fixes that were
rejected by some or all participants to understand the reasons.

\(F2\) was rejected by our participants because the region of the commit
that triggered a match is a generated code. Although this generated code
was pushed into the repositories as part of bug fixing commit, the root
cause of the bug lies in the code generator itself. Our proposed fix
suggests to update the generated code. Because the proposed fix did not
apply directly to the the question we ask our reviewers was \emph{``Is
the proposed fix applicable in the given situation?''} they rejected it.
In this occurrence, the proposed fix was not applicable.

\(F4\) was accepted by two reviewers and marked as unsure by the other
participants. We believe that this was due the lack of context
surrounding the proposed fix. The participants were unable to determine
if the fix was applicable or not without knowing what the original
intent of the buggy commit was. In our review session, we only provided
the reviewers with the regions of the commits that matched existing
commits and not the full commit. Full commits can be quite lengthy as
they can contain asset descriptions and generated code, in addition to
the actual code. In this occurrence, the full context of the commit
might have helped our reviewers to decide if the \(F4\) was applicable
or not. \(F5\) and \(F7\) were classified as unsure by all our
participants for the same reasons.

\(F8\) was rejected by four of participants and accepted by two. The
participant argued that the proposed fix was more a refactoring
opportunity than an actual fix.

\(F12\) was marked as unsure by all the reviewers because the code had
to do with a subsystem that is maintained by another team and the
participants felt that it was out of scope of this session focusing on
their system.

After the session, we asked the participants two additional questions:
\emph{Will you use CLEVER in the future?} and \emph{What aspects of
CLEVER need to be improved?}

The participants answered the first question positively. They all agreed
that CLEVER could be a good tool for intercepting risky commits, and
hence improving the quality assurance process. For the second question,
the participants expressed concerns about the context surrounding the
buggy commits and the fixes. While displaying the entire commits is not
a solution, according to the participants, some context might be
inferred from the commit messages and the issues associated with the
fixes in the bug tracking systems. The second limitation of CLEVER is
its inability to deal with generated code. At this point, CLEVER points
towards the generated code rather than the code generator. These aspects
of CLEVER needs to be improved.

\section{Discussion}\label{sec:threats}

In this section, we propose a discussion on limitations and threats to
validity.

\subsection{Limitations}\label{limitations}

We identified two main limitations of our approach, CLEVER, which
require further studies.

CLEVER is designed to work on multiple related systems. Applying CLEVER
on a single system will most likely be less effective. The the
two-phases classification process of CLEVER would be hindered by the
fact that it is unlikely to have a large number of similar bugs within
the same system. For single systems, we recommend the use of
metric-based models. A metric-based solution, however, may turn to be
ineffective when applied across systems because of the difficulty
associated with identifying common thresholds that are applicable to a
wide range of systems.

The second limitation we identified has to do with the fact that CLEVER
is designed to work with Ubisoft systems. Ubisoft uses C\#, C, C++, Java
and other internally developed languages. It is however common to have
other languages used in an environment with many inter-related systems.
We intend to extend CLEVER to process commits from other languages as
well.

\subsection{Threats to Validity}\label{threats-to-validity}

The selection of target systems is one of the common threats to validity
for approaches aiming to improve the analysis of software systems. It is
possible that the selected programs share common properties that we are
not aware of and therefore, invalidate our results. Because of the
industrial nature of this study, we had to work with the systems
developed by the company.

The programs we used in this study are all based on the C\#, C, C++ and
Java programming languages. This can limit the generalization of the
results to projects written in other languages, especially that the main
component of CLEVER is based on code clone matching.

Finally, part of the analysis of the CLEVER proposed fixes that we did
was based on manual comparisons of the CLEVER fixes with those proposed
by developers with a focus group composed of experienced engineers and
software architects. Although, we exercised great care in analysing all
the fixes, we may have misunderstood some aspects of the commits.

In conclusion, internal and external validity have both been minimized
by choosing a set of 12 different systems, using input data that can be
found in any programming languages and version systems (commits and
changesets).

\section{Conclusion}\label{sec:conclusion}

In this paper, we presented CLEVER (Combining Levels of Bug Prevention
and Resolution Techniques), an approach that detects risky commits
(i.e., a commit that is likely to introduce a bug) with an average of
79.10\% precision and a 65.61\% recall. CLEVER combines code metrics,
clone detection techniques and project dependency analysis to detect
risky commits within and across projects. CLEVER operates at
commit-time, i.e., before the commits reach the central code repository.
Also, because it relies on code comparison, CLEVER does not only detect
risky commits but also makes recommendations to developers on how to fix
them. We believe that this makes CLEVER a practical approach for
preventing bugs and proposing corrective measures that integrate well
with the developer's workflow through the commit mechanism.

As a future work, we want to build a feedback loop between the users and
the clusters of known buggy commits and their fixes. If a fix is never
used by the end-users, then we could remove it from the clusters and
improve our accuracy. We also intend to improve CLEVER to deal with
generated code. We also want to improve the fixes proposed by CLEVER to
add contextual information to help developers better assess the
applicability of the fixes.

\section{Reproduction Package}\label{reproduction-package}

For security and confidentiality reasons we cannot provide a
reproduction package that will inevitably involve Ubisoft's copyrighted
source code. However, the CLEVER source code is in the process of being
open-sourced and will be soon available at
https://github.com/ubisoftinc.

\begin{acks}
We are thankful to Olivier Pomarez, Yves Jacquier, Nicolas Fleury, Alain Bedel, Mark Besner, David Punset, Paul Vlasie, Cyrille Gauclin, Luc Bouchard, Chadi Lebbos and Florent Jousset, Anthony Brien, Thierry Jouin and Jean-Pierre Nolin from Ubisoft for their participations in validating CLEVER hypothesis, efficiency and proposed fixes.
\end{acks}

\section*{References}

\setlength{\parindent}{0pt}

\hypertarget{refs}{}
\hypertarget{ref-Bhattacharya2011}{}
{[}1{]} Bhattacharya, P. and Neamtiu, I. 2011. Bug-fix time prediction
models: can we do better? \emph{Proceeding of the international
conference on mining software repositories} (New York, New York, USA,
May 2011), 207--210.

\hypertarget{ref-Briand1999a}{}
{[}2{]} Briand, L. et al. 1999. A unified framework for coupling
measurement in object-oriented systems. \emph{IEEE Transactions on
Software Engineering}. 25, 1 (1999), 91--121.
DOI:\url{https://doi.org/10.1109/32.748920}.

\hypertarget{ref-bultena1998eades}{}
{[}3{]} Bultena, B. and Ruskey, F. 1998. An Eades-McKay algorithm for
well-formed parentheses strings. \emph{Information Processing Letters}.
68, 5 (1998), 255--259.

\hypertarget{ref-Chen2014}{}
{[}4{]} Chen, T.-h. et al. 2014. An Empirical Study of Dormant Bugs
Categories and Subject Descriptors. \emph{Proceedings of the
international conference on mining software repository} (2014), 82--91.

\hypertarget{ref-Chidamber1994}{}
{[}5{]} Chidamber, S. and Kemerer, C. 1994. A metrics suite for object
oriented design. \emph{IEEE Transactions on Software Engineering}. 20, 6
(Jun. 1994), 476--493. DOI:\url{https://doi.org/10.1109/32.295895}.

\hypertarget{ref-Cordy2011}{}
{[}6{]} Cordy, J.R. and Roy, C.K. 2011. The NiCad Clone Detector.
\emph{Proceedings of the international conference on program
comprehension} (Jun. 2011), 219--220.

\hypertarget{ref-Dallmeier}{}
{[}7{]} Dallmeier, V. et al. 2009. Generating Fixes from Object Behavior
Anomalies. \emph{Proceedings of the international conference on
automated software engineering} (2009), 550--554.

\hypertarget{ref-StephaneDucasse}{}
{[}8{]} Ducasse, S. et al. 1999. A Language Independent Approach for
Detecting Duplicated Code. \emph{Proceedings of the international
conference on software maintenance} (1999), 109--118.

\hypertarget{ref-ElEmam2001}{}
{[}9{]} El Emam, K. et al. 2001. The prediction of faulty classes using
object-oriented design metrics. \emph{Journal of Systems and Software}.
56, 1 (Feb. 2001), 63--75.
DOI:\url{https://doi.org/10.1016/S0164-1212(00)00086-8}.

\hypertarget{ref-Girvan2002}{}
{[}10{]} Girvan, M. and Newman, M.E.J. 2002. Community structure in
social and biological networks. \emph{Proceedings of the National
Academy of Sciences}. 99, 12 (Jun. 2002), 7821--7826.
DOI:\url{https://doi.org/10.1073/pnas.122653799}.

\hypertarget{ref-Gyimothy2005}{}
{[}11{]} Gyimothy, T. et al. 2005. Empirical validation of
object-oriented metrics on open source software for fault prediction.
\emph{IEEE Transactions on Software Engineering}. 31, 10 (Oct. 2005),
897--910. DOI:\url{https://doi.org/10.1109/TSE.2005.112}.

\hypertarget{ref-Hassan2005}{}
{[}12{]} Hassan, A. and Holt, R. 2005. The top ten list: dynamic fault
prediction. \emph{Proceedings of the international conference on
software maintenance} (2005), 263--272.

\hypertarget{ref-Hassan2009}{}
{[}13{]} Hassan, A.E. 2009. Predicting faults using the complexity of
code changes. \emph{Proceedings of the international conference on
software engineering} (May 2009), 78--88.

\hypertarget{ref-Hindle2008}{}
{[}14{]} Hindle, A. et al. 2008. What do large commits tell us?: a
taxonomical study of large commits. \emph{Proceedings of the
international workshop on mining software repositories} (New York, New
York, USA, 2008), 99--108.

\hypertarget{ref-Hunt1977}{}
{[}15{]} Hunt, J.W. and Szymanski, T.G. 1977. A fast algorithm for
computing longest common subsequences. \emph{Communications of the ACM}.
20, 5 (May 1977), 350--353.
DOI:\url{https://doi.org/10.1145/359581.359603}.

\hypertarget{ref-Johnson1993}{}
{[}16{]} Johnson, J.H. 1993. Identifying redundancy in source code using
fingerprints. \emph{Proceedings of the conference of the centre for
advanced studies on collaborative research} (Oct. 1993), 171--183.

\hypertarget{ref-Johnson1994}{}
{[}17{]} Johnson, J.H. 1994. Visualizing textual redundancy in legacy
source. \emph{Proceedings of the conference of the centre for advanced
studies on collaborative research} (Oct. 1994), 32.

\hypertarget{ref-Kamei2013a}{}
{[}18{]} Kamei, Y. et al. 2013. A large-scale empirical study of
just-in-time quality assurance. \emph{IEEE Transactions on Software
Engineering}. 39, 6 (Jun. 2013), 757--773.
DOI:\url{https://doi.org/10.1109/TSE.2012.70}.

\hypertarget{ref-Kamei2013}{}
{[}19{]} Kamei, Y. et al. 2013. A large-scale empirical study of
just-in-time quality assurance. \emph{IEEE Transactions on Software
Engineering}. 39, 6 (Jun. 2013), 757--773.
DOI:\url{https://doi.org/10.1109/TSE.2012.70}.

\hypertarget{ref-Kamei2013b}{}
{[}20{]} Kamei, Y. et al. 2013. A large-scale empirical study of
just-in-time quality assurance. \emph{IEEE Transactions on Software
Engineering}. 39, 6 (Jun. 2013), 757--773.
DOI:\url{https://doi.org/10.1109/TSE.2012.70}.

\hypertarget{ref-CoryKapser}{}
{[}21{]} Kapser, C. and Godfrey, M.W. 2003. Toward a Taxonomy of Clones
in Source Code: A Case Study. \emph{International workshop on evolution
of large scale industrial software architectures} (2003), 67--78.

\hypertarget{ref-Kim2013}{}
{[}22{]} Kim, D. et al. 2013. Automatic patch generation learned from
human-written patches. \emph{Proceedings of the international conference
on software engineering} (May 2013), 802--811.

\hypertarget{ref-Kim2006c}{}
{[}23{]} Kim, S. et al. 2006. Automatic Identification of
Bug-Introducing Changes. \emph{Proceedings of the international
conference on automated software engineering} (2006), 81--90.

\hypertarget{ref-Kim2007a}{}
{[}24{]} Kim, S. et al. 2007. Predicting Faults from Cached History.
\emph{Proceedings of the international conference on software
engineering} (May 2007), 489--498.

\hypertarget{ref-Kpodjedo2010}{}
{[}25{]} Kpodjedo, S. et al. 2010. Design evolution metrics for defect
prediction in object oriented systems. \emph{Empirical Software
Engineering}. 16, 1 (Dec. 2010), 141--175.
DOI:\url{https://doi.org/10.1007/s10664-010-9151-7}.

\hypertarget{ref-le2012systematic}{}
{[}26{]} Le Goues, C. et al. 2012. A systematic study of automated
program repair: Fixing 55 out of 105 bugs for \$8 each.
\emph{Proceedings of the international conference on software
engineering} (2012), 3--13.

\hypertarget{ref-le2015should}{}
{[}27{]} Le, X.-B.D. et al. 2015. Should fixing these failures be
delegated to automated program repair? \emph{Proceedings of the
international symposium on software reliability engineering} (2015),
427--437.

\hypertarget{ref-Lee2011a}{}
{[}28{]} Lee, T. et al. 2011. Micro interaction metrics for defect
prediction. \emph{Proceedings of the european conference on foundations
of software engineering} (New York, New York, USA, 2011), 311--231.

\hypertarget{ref-Manber1994}{}
{[}29{]} Manber, U. 1994. Finding similar files in a large file system.
\emph{Proceedings of the usenix winter} (Jan. 1994), 1--10.

\hypertarget{ref-Marcus}{}
{[}30{]} Marcus, A. and Maletic, J. 2001. Identification of high-level
concept clones in source code. \emph{Proceedings international
conference on automated software engineering} (2001), 107--114.

\hypertarget{ref-Nagappan2005}{}
{[}31{]} Nagappan, N. and Ball, T. 2005. Static analysis tools as early
indicators of pre-release defect density. \emph{Proceedings of the
international conference on software engineering} (New York, New York,
USA, May 2005), 580--586.

\hypertarget{ref-Nagappan}{}
{[}32{]} Nagappan, N. and Ball, T. 2005. Use of relative code churn
measures to predict system defect density. \emph{Proceedings of the
international conference on software engineering} (2005), 284--292.

\hypertarget{ref-Nagappan2006}{}
{[}33{]} Nagappan, N. et al. 2006. Mining metrics to predict component
failures. \emph{Proceeding of the international conference on software
engineering} (New York, New York, USA, May 2006), 452--461.

\hypertarget{ref-Nam2013}{}
{[}34{]} Nam, J. et al. 2013. Transfer defect learning.
\emph{Proceedings of the international conference on software
engineering} (May 2013), 382--391.

\hypertarget{ref-Newman2004}{}
{[}35{]} Newman, M.E.J. and Girvan, M. 2004. Finding and evaluating
community structure in networks. \emph{Physical Review E}. 69, 2 (Feb.
2004), 026113. DOI:\url{https://doi.org/10.1103/PhysRevE.69.026113}.

\hypertarget{ref-Ostrand2005}{}
{[}36{]} Ostrand, T. et al. 2005. Predicting the location and number of
faults in large software systems. \emph{IEEE Transactions on Software
Engineering}. 31, 4 (Apr. 2005), 340--355.
DOI:\url{https://doi.org/10.1109/TSE.2005.49}.

\hypertarget{ref-Pan2008}{}
{[}37{]} Pan, K. et al. 2008. Toward an understanding of bug fix
patterns. \emph{Empirical Software Engineering}. 14, 3 (Aug. 2008),
286--315. DOI:\url{https://doi.org/10.1007/s10664-008-9077-5}.

\hypertarget{ref-rahman2013}{}
{[}38{]} Rahman, F. and Devanbu, P. 2013. How, and why, process metrics
are better. \emph{Proceedings of the international conference on
software engineering} (2013), 432--441.

\hypertarget{ref-Rosen2015}{}
{[}39{]} Rosen, C. et al. 2015. Commit guru: analytics and risk
prediction of software commits. \emph{Proceedings of the joint meeting
on foundations of software engineering} (New York, New York, USA, Aug.
2015), 966--969.

\hypertarget{ref-Iss2009}{}
{[}40{]} Roy, C.K. 2009. \emph{Detection and Analysis of Near-Miss
Software Clones}. Queen's University.

\hypertarget{ref-Roy2008}{}
{[}41{]} Roy, C.K. and Cordy, J.R. 2008. An Empirical Study of Function
Clones in Open Source Software. \emph{Proceedings of the working
conference on reverse engineering} (Oct. 2008), 81--90.

\hypertarget{ref-Schein2002}{}
{[}42{]} Schein, A.I. et al. 2002. Methods and metrics for cold-start
recommendations. \emph{Proceedings of the annual international onference
on research and development in information retrieval} (New York, New
York, USA, 2002), 253.

\hypertarget{ref-Shivaji2013}{}
{[}43{]} Shivaji, S. et al. 2013. Reducing Features to Improve Code
Change-Based Bug Prediction. \emph{IEEE Transactions on Software
Engineering}. 39, 4 (2013), 552--569.

\hypertarget{ref-Subramanyam2003}{}
{[}44{]} Subramanyam, R. and Krishnan, M. 2003. Empirical analysis of CK
metrics for object-oriented design complexity: implications for software
defects. \emph{IEEE Transactions on Software Engineering}. 29, 4 (Apr.
2003), 297--310. DOI:\url{https://doi.org/10.1109/TSE.2003.1191795}.

\hypertarget{ref-SunghunKim2008}{}
{[}45{]} Sunghun Kim, S. et al. 2008. Classifying Software Changes:
Clean or Buggy? \emph{IEEE Transactions on Software Engineering}. 34, 2
(Mar. 2008), 181--196. DOI:\url{https://doi.org/10.1109/TSE.2007.70773}.

\hypertarget{ref-tao2014automatically}{}
{[}46{]} Tao, Y. et al. 2014. Automatically generated patches as
debugging aids: a human study. \emph{Proceedings of the international
symposium on foundations of software engineering} (2014), 64--74.

\hypertarget{ref-Wettel2005}{}
{[}47{]} Wettel, R. and Marinescu, R. 2005. Archeology of code
duplication: recovering duplication chains from small duplication
fragments. \emph{Proceedings of the seventh international symposium on
symbolic and numeric algorithms for scientific computing} (2005),
63--71.

\hypertarget{ref-Zimmermann2008}{}
{[}48{]} Zimmermann, T. and Nagappan, N. 2008. Predicting defects using
network analysis on dependency graphs. \emph{Proceedings of the
international conference on software engineering} (New York, New York,
USA, May 2008), 531.

\hypertarget{ref-Zimmermann2007}{}
{[}49{]} Zimmermann, T. et al. 2007. Predicting Defects for Eclipse.
\emph{Proceedings of the international workshop on predictor models in
software engineering} (May 2007), 9.

\bibliographystyle{ACM-Reference-Format}
\bibliography{config/library.bib}


\end{document}
